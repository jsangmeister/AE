\documentclass[ngerman,aspectratio=169,10pt]{beamer}

\usetheme[progressbar=frametitle]{metropolis}
\usepackage{appendixnumberbeamer}

\graphicspath{{./graphics/}}

\usepackage{booktabs}
\usepackage{xspace}
\usepackage{amsmath}
\usepackage{amssymb}
\usepackage{amsthm}
\usepackage{xfrac}
\usepackage{listings}
\lstset{
	basicstyle=\ttfamily,
	showstringspaces=false,
	tabsize=4,
	upquote=true,
}

\title{TSP Varianten}
% \subtitle{}
\date{14. Dezember 2020}
\author{Finn Stutzenstein, Levin Nemesch, Joshua Sangmeister}
\institute{Algorithm Engineering - Übung 3}
\titlegraphic{
    \hfill\includegraphics[height=1.5cm]{unilogo.pdf}\\
    \hspace*{8.3cm} \textsc{AG Theoretische Informatik}
}

\begin{document}

\maketitle

\begin{frame}{TSP aus der Vorlesung}
	\textbf{Gegeben}: $G=(V,E)$ ungerichtet, $w_e\geq0$ Kantengewichte\\
	\textbf{Gesucht}: Rundtour $E'\subseteq E$, sodass $\sum_{e\in E'}w_e$ minimal ist
	
	\begin{align*}
	\min && \sum_{e\in E}w_ex_e&&\\
	\text{s.t.} && \sum_{e\in \delta(v)}x_e &=2 &\forall v\in V\\
	&& \sum_{e\in \delta(W)}x_e &\geq2 &\forall \emptyset\neq W\subset V\\
	&& x_e &\in\{0,1\} &\forall e\in E
	\end{align*}
\end{frame}

\begin{frame}{Asymmetrisches TSP (ATSP)\footnote{Dantzig G., Fulkerson D., Johnson S.: Solutions of a large-scale traveling-salesman problem. (1954) \emph{Operation Research} 2(4):393–410}}
	% https://link.springer.com/article/10.1007/s13676-012-0010-0
	\textbf{Gegeben}: $G=(V,A)$ gerichtet, $w_{ij}\geq0$ Kantengewichte (i.A. $w_{uv}\neq w_{vu}$, keine Selfloops: $w_{ii}=+\infty$)\\
	\textbf{Gesucht}: Gerichtete Rundtour $A'\subseteq A$, sodass $\sum_{(i,j)\in A'}w_{ij}$ minimal ist
	
	\textbf{ILP}:
	\begin{itemize}
		\item Dantzig–Fulkerson–Johnson Formulierung
		\item $x_{ij}=1\quad\Leftrightarrow\quad$Kante $(i,j)$ in der Lösung
	\end{itemize}
\end{frame}
\begin{frame}{Asymmetrisches TSP (ATSP)}
	
	\begin{align*}
	\min && \sum_{i\in V}\sum_{j\in V}w_{ij}x_{ij}&&\\
	\text{s.t.} && \sum_{i\in V}x_{ij} &=1 &\forall j\in V\\
	&& \sum_{j\in V}x_{ij} &=1 &\forall i\in V\\
	&& \sum_{i\in W}\sum_{j\in W}x_{ij} &\leq|W|-1 &\forall \emptyset\neq W\subset V\\
	&& x_{ij} &\in\{0,1\} &\forall i,j\in V
	\end{align*}
\end{frame}


\begin{frame}{Bottleneck TSP\footnote{R. S. Garfinkel und K. C. Gilbert. The Bottleneck Traveling Salesman Problem: Algorithms and Probabilistic Analysis. \emph{Journal of the ACM (JACM)} 25.3 (1978), S. 435–448.}}
	\textbf{Gegeben}: $G=(V,E)$ ungerichtet, $w_e\geq0$ Kantengewichte\\
	\textbf{Gesucht}: Rundtour $E'\subseteq E$, sodass $\max\limits_{e\in E'}(w_e)$ minimal ist
	
	\textbf{ILP}:
	\begin{itemize}
		\item $y\in\mathbb{Z}$ zum Messen der teuersten Kante: $y\geq w_e x_e \forall e\in E$
	\end{itemize}
\end{frame}
\begin{frame}{Bottleneck TSP}
	\begin{align*}
	\min && y &&\\
	\text{s.t.} && \sum_{e\in \delta(v)}x_e &=2 &\forall v\in V\\
	&& \sum_{e\in \delta(W)}x_e &\geq2 &\forall \emptyset\neq W\subset V\\
	&& y - w_e x_e &\geq0 &\forall e \in E\\
	&& x_e &\in\{0,1\} &\forall e\in E\\
	&& y &\in\mathbb{Z} &
	\end{align*}
\end{frame}

\begin{frame}{Equitable TSP\footnote{J. Kinable, B. Smeulders, E. Delcour und F. C. Spieksma. Exact algorithms for the equitable traveling salesman problem. \emph{European Journal of Operational Research} 261.2 (2017), S. 475 –485.}}
	\textbf{Gegeben}: $G=(V,E)$ ungerichtet, $w_e\geq0$ Kantengewichte, $n$ gerade.\\
	\textbf{Gesucht}: Zwei perfekte Matchings $M_1,M_2\subset E$, sodass
	\begin{enumerate}
		\item die Vereinigung beider Matchings einen Hamiltonkreis in $G$ bilden und
		\item $|c(M_1)-c(M_2)|$ mit $c(M)=\sum_{e\in M}w_e$ minimal ist.
	\end{enumerate}
	
	\textbf{ILP}:
	\begin{itemize}
		\item $\mathcal{M}$: Menge aller perfekten Matchings von $G$
		\item $y_M^1=1~(y_M^2=1)\quad\Leftrightarrow\quad$: Matching $M\in\mathcal{M}$ wird als erstes (zweites) Matching gewählt
	\end{itemize}
\end{frame}
\begin{frame}{Equitable TSP}
	\begin{align*}
	\min    && \left|\sum_{M\in\mathcal{M}}c(M)y_M^1-\sum_{M\in\mathcal{M}}c(M)y_M^2\right|\\
	\text{s.t.} && \sum_{M\in\mathcal{M}}y_M^1 &=1&&\\
	&& \sum_{M\in\mathcal{M}}y_M^2 &=1&&\\
	&& \sum_{M\in\mathcal{M}:e\in M}y_M^1+\sum_{M\in\mathcal{M}:e\in M}y_M^2 &\leq1 &&\forall e\in E\\
	&& \sum_{e\in\delta(W)}\left(\sum_{M\in\mathcal{M}:e\in M}y_M^1+\sum_{M\in\mathcal{M}:e\in M}y_M^2\right) &\geq 2 &&\forall W\subset V, |S| \geq 3\\
	&& y_M^1 &\in\{0,1\} &&\forall M\in\mathcal{M}\\
	&& y_M^2 &\in\{0,1\} &&\forall M\in\mathcal{M}
	\end{align*}
\end{frame}

\begin{frame}{Multiples TSP\footnote{Bektas, T.: The multiple traveling salesman problem: an overview of formulations and solution procedures. \emph{Omega}, 34(3) (2006), pp. 209–219.}}
	% https://profs.info.uaic.ro/~mtsplib/Standard%20SD-MTSP.pdf
	\textbf{Gegeben}: $G=(V,A)$, $w_{ij}\geq0$ Kantengewichte, Depot $d\in V$, $m$ Verkäufer\\
	\textbf{Gesucht}: $m$ Touren mit jeweils mindestens 3 Knoten, die alle $d$ beinhalten. Die Gesamtwegkosten müssen minimal sein. Jeder Knoten (außer $d$) muss in genau einer Tour sein.\\
	\emph{Anmerkung}: Single Depot-Variante; Ein Depot pro Verkäufer möglich (Multi Depot-Variante)\\

	\textbf{ILP}:
	\begin{itemize}
		\item $x_{ij}=1\quad\Leftrightarrow\quad$Kante $(i,j)$ in der Lösung
		\item $u_i$: Anzahl der besuchten Knoten vom Depot zu $i$ für jeden Verkäufer (also Position von $i$ in der Tour)
		\item $V'=V\setminus\{1\}$, Knoten 1 ist das Depot
	\end{itemize}
\end{frame}
\begin{frame}{Multiples TSP}
	\begin{align*}
	\min && \sum_{i\in V}\sum_{j\in V}w_{ij}x_{ij}&&\\
	\text{s.t.} && \sum_{j\in V'}x_{1j}=\sum_{i\in V'}x_{i1}&=m&\\
	&& \sum_{i\in V'}x_{ij} &=1 &\forall j\in V'\\
	&& \sum_{j\in V'}x_{ij} &=1 &\forall i\in V'\\
	&& x_{1i}+x_{i1} &\leq1 &\forall i\in V'\\
	&& u_i-u_j+(n-m)x_{ij} &\leq n-m-1 &\forall i\neq j\in V'\\
	&& x_{ij} &\in\{0,1\} &\forall i,j\in V
	\end{align*}
\end{frame}

\begin{frame}{Prizecollecting TSP\footnote{Vansteenwegen P., Gunawan A.: Orienteering Problems. \emph{EURO Advanced Tutorials on Operational Research} (2019)}}
	% https://link.springer.com/book/10.1007/978-3-030-29746-6
	\textbf{Gegeben}: $G=(V,E)$ ungerichtet, $w_e\geq0$ Kantengewichte, $p_i$ Knotenprofite, $P_{min}$ zu erreichenden Minimalprofit, $d\in V$ Depot\\
	\textbf{Gesucht}: Rundtour $E'\subseteq E$ über Auswahl an Knoten $V'\subseteq V$, sodass $\sum_{e\in E'}w_e$ minimal ist und $\sum_{v\in V'}p_v\geq P_{min}$ gilt
	
	\textbf{ILP}:
	\begin{itemize}
		\item Ordnungsvariablen $x_{ij}=1\quad\Leftrightarrow\quad$Nach Knoten $i$ folgt Knoten $j$ ($i\rightarrow j$)
		\item $u_i$: Position von Knoten $i$ in der Tour
		\item $V'=V\setminus\{1\}$, Knoten 1 ist das Depot
	\end{itemize}
\end{frame}

\begin{frame}{Prizecollecting TSP}
	\begin{align*}
	\min && \sum_{i\in V}\sum_{j\in V}w_{ij}x_{ij}&&\\
	\text{s.t.} && \sum_{j\in V'}x_{1j}=\sum_{i\in V'}x_{i1}&=1&\\
	&& \sum_{i\in V}x_{ik} = \sum_{j\in V}x_{kj}&\leq1 &\forall k\in V'\\
	&& \sum_{i\in V}\sum_{j\in V}p_ix_{ij} &\geq P_{min} &\\
	&& u_i-u_j+1&\leq(n-1)(1-x_{ij})&\forall i,j\in V'\\
	&& x_{ij} &\in\{0,1\} &\forall i,j\in V
	\end{align*}
\end{frame}

\begin{frame}{Multi-Objective TSP\footnote{T. Lust und J. Teghem. The Multiobjective Traveling Salesman Problem: A Survey and a New Approach. \emph{Advances in Multi-Objective Nature Inspired Computing} (2010), S. 119–141.}}
	\textbf{Gegeben}: $G=(V,E)$ ungerichtet, $d\geq1$, $w_e\in{\mathbb{N}^d}$ Kantengewichte\\
	\textbf{Gesucht}: Rundtouren $E_\text{nd}=\{E'\subseteq E\}$, sodass $\sum_{e\in E'}w_e,E'\in E_\text{nd}$ nicht-dominiert ist

	\textbf{ILP}:
	\begin{itemize}
		\item Constraints wie normales TSP, Zielfunktion mehrkriteriell\\ (Streng genommen kein ILP)
	\end{itemize}
	\begin{align*}
		{\min}_\preceq && \sum_{e\in E}w_ex_e&&\in\mathbb{N}^d\\
		\text{s.t.} && \sum_{e\in \delta(v)}x_e &=2 &\forall v\in V\\
		&& \sum_{e\in \delta(W)}x_e &\geq2 &\forall \emptyset\neq W\subset V\\
		&& x_e &\in\{0,1\} &\forall e\in E
	\end{align*}
\end{frame}

\end{document}