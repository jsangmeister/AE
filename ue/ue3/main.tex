\documentclass[ngerman,aspectratio=169,10pt]{beamer}

\usetheme[progressbar=frametitle]{metropolis}
\usepackage{appendixnumberbeamer}

\graphicspath{{./graphics/}}

\usepackage{booktabs}
\usepackage{xspace}
\usepackage{amsmath}
\usepackage{amssymb}
\usepackage{amsthm}
\usepackage{xfrac}
\usepackage{listings}
\lstset{
	basicstyle=\ttfamily,
	showstringspaces=false,
	tabsize=4,
	upquote=true,
}

\title{TSP Varianten}
% \subtitle{}
\date{14. Dezember 2020}
\author{Finn Stutzenstein, Levin Nemesch, Joshua Sangmeister}
\institute{Algorithm Engineering - Übung 3}
\titlegraphic{
    \hfill\includegraphics[height=1.5cm]{unilogo.pdf}\\
    \hspace*{8.3cm} \textsc{AG Theoretische Informatik}
}

\begin{document}

\maketitle

\begin{frame}
	\begin{itemize}
		\item TSP
		\item multiples TSP
		\item asymmetrisches TSP
		\item prizecollecting TSP
	\end{itemize}
\end{frame}

\begin{frame}{TSP}
	\textbf{Gegeben}: $G=(V,E)$ ungerichtet, $w_e\geq0$ Kantengewichte\\
	\textbf{Gesucht}: Rundtour $E'\subseteq E$, sodass $\sum_{e\in E'}w_e$ minimal ist
	
	\begin{align*}
	\min && \sum_{e\in E}x_ew_e&&\\
	\text{s.t.} && \sum_{e\in \delta(v)}x_e &=2 &\forall v\in V\\
	&& \sum_{e\in \delta(W)}x_e &\geq2 &\forall \emptyset\neq W\subset V\\
	&& x_e &\in\{0,1\} &\forall e\in E
	\end{align*}
\end{frame}

\begin{frame}{Multiples TSP (Relaxation vom VRP)}
	\textbf{Gegeben}: $G=(V,E)$, $w_e\geq0$ Kantengewichte, Depot $d\in V$, $m$ Verkäufer\\
	\textbf{Gesucht}: $m$ Touren, die alle $d$ beinhalten mit minimalen Gesamtwegekosten. Dabei muss jeder Knoten (außer $d$) in genau einer Tour sein.
\end{frame}

\begin{frame}{Asymmetrisches TSP}
	\textbf{Gegeben}: $G=(V,A)$ gerichtet, $w_e\geq0$ Kantengewichte (i.A. $w_{uv}\neq w_{vu}$)\\
	\textbf{Gesucht}: Gerichtete Rundtour $A'\subseteq A$, sodass $\sum_{a\in A'}w_a$ minimal ist
\end{frame}

\begin{frame}{Prizecollecting TSP}
	\textbf{Gegeben}: $G=(V,E)$
\end{frame}

\begin{frame}{Das kann als vorlage zum erstellen von LPs genutzt werden}
   \begin{align}
   \min    && \left|\sum_{M\in\mathcal{M}^b}c^b(M)y_M^b-\sum_{M\in\mathcal{M}^r}c^r(M)y_M^r\right| \tag{Z}\label{eq:z}\\
   \text{s.t.} && \sum_{M\in\mathcal{M}^b}y_M^b &=1&& \tag{C1}\label{eq:c1}\\
   && \sum_{M\in\mathcal{M}^r}y_M^r &=1&& \tag{C2}\label{eq:c2}\\
   && \sum_{M\in\mathcal{M}^b:e\in M}y_M^b+\sum_{M\in\mathcal{M}^r:e\in M}y_M^r &\leq1 &&\forall e\in E_b\cap E_r \tag{C3}\label{eq:c3}\\
   && \sum_{e\in\delta(S)}\left(\sum_{M\in\mathcal{M}^b:e\in M}y_M^b+\sum_{M\in\mathcal{M}^r:e\in M}y_M^r\right) &\geq 2 &&\forall S\subset V, |S| \geq 3 \tag{C4}\label{eq:c4}\\[0.5cm]
   && y_M^b &\in\{0,1\} &&\forall M\in\mathcal{M}^b \tag{C5}\label{eq:c5}\\
   && y_M^r &\in\{0,1\} &&\forall M\in\mathcal{M}^r \tag{C6}\label{eq:c6}
   \end{align}
\end{frame}


\end{document}